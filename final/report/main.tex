\documentclass[12pt,a4paper,titlepage]{article}

\usepackage[spanish]{babel}
\usepackage[utf8]{inputenc}
\usepackage{pgfplotstable,filecontents}
\pgfplotsset{compat=1.9}% supress warning
\usepackage{geometry}
\usepackage{layout}

\geometry{
  left=2.5cm,
  right=2.5cm,
  top=3.5cm,
  bottom=3cm
}

\usepackage{setspace}
\doublespacing

\usepackage{hyperref}

\title{
  \large{Programación Distribuída sobre Grandes Volúmenes de Datos}\\
  \Large{Análisis de Proyectos de Ley de la Honorable Cámara de Diputados de la Nación Argentina}
}

\author{Biedma, Luis A.\\         \texttt{lbiedma@famaf.unc.edu.ar}
        \and
        Cardellino, Cristian A.\\ \texttt{ccardellino@famaf.unc.edu.ar}}

\date{Miércoles, 21 de Diciembre del 2016}

\begin{document}
  \maketitle

  \section{Introducción}\label{section:introduction}

  El presente informe presenta el trabajo de análisis realizado sobre los
  proyectos de ley propuestos por la Honorable Cámara de Diputados de la Nación
  Argentina\footnote{\url{http://www.hcdn.gob.ar/}} (HCDN), entre la primera
  sesión ordinaria del 1 Marzo del 2011 y el 8 de Octubre de 2016.

  El objetivo de este trabajo es poder realizar un estudio, si bien ligeramente
  superficial, sobre el tipo de proyectos que se presentan en la HCDN, cuáles
  son los que terminan por sancionarse, y qué partidos, distritos y
  legisladores suelen estar involucrados en los mismos.

  La motivación surge a partir de una serie de publicaciones de una usuaria de
  reddit Argentina\footnote{\url{http://www.reddit.com/r/argentina}}, donde se
  hace un resumen semana a semana de las actividades de la totalidad del
  Congreso de la Nación Argentina.  Se resolvió hacer el trabajo sólo sobre la
  HCDN y no sobre la totalidad del congreso por cuestiones prácticas, debido a
  que sólo el HCDN tiene una estructura en su sitio web suficientemente
  uniforme como para poder conseguir los datos a analizar.

  El proyecto se divide, a grandes rasgos, en 4 etapas principales: obtención
  de datos; análisis estadístico de los proyectos; modelado de los temas ({\em
  topic modelling}) de los textos de los proyectos de ley y subsecuente
  análisis estadístico de esta información; y entrenamiento de un clasificador
  para predecir si un proyecto será eventualmente sancionado.

  Este informe se estructura de la siguiente manera: la
  Sección~\ref{section:data} presenta una descripción de los datos obtenidos,
  el lugar del cuál se obtuvieron y la manera de obtenerlos, también explica
  las decisiones que se tomaron a la hora de seleccionar los datos. La
  Sección~\ref{section:statistics} presenta el análisis estadístico general
  sobre los proyectos (sin tener en cuenta necesariamente su contenido). La
  Sección~\ref{section:lda} explica el proceso de modelado de temas mediante la
  técnica de Latent Dirichlet Allocation, los distintos experimentos que se
  realizaron y un análisis estadístico entre los proyectos y los temas
  obtenidos.
  %%%% TODO: Ver si llegamos
  La Sección~\ref{section:classification} describe el proceso seguido para
  conseguir un clasificador de proyectos para predecir si estos serán o no
  sancionados.
  %%%%
  Finalmente, en la Sección\ref{section:conclusions} se enumeran las
  conclusiones del trabajo y se enlistan aquellas cosas que quedan para posible
  trabajo futuro.

  \section{Obtención de datos}\label{section:data}

  Los datos fueron obtenidos mediante el sitio web oficial de la Honorable Cámara
  de Diputados de la Nación.  Los proyectos del sitio se pueden obtener
  mediante la dirección
  \url{http://www.hcdn.gob.ar/proyectos/proyectoTP.jsp?exp=EXPEDIENTE}, donde
  {\tt EXPEDIENTE} es el identificador del proyecto y se compone de tres partes
  separadas por un {\em guión} (-): el primero es un número de al menos 4
  cifras (que se rellena con ceros en caso de ser menor que 1000), que
  representa la identificación del proyecto en un período legislativo; le sigue
  una sigla que representa la cámara o entidad que presentó el proyecto: D para
  {\em diputados}, S para {\em senado}, PE para {\em Poder Ejecutivo} y JGM
  para {\em Jefatura Gabinete Ministros}.
  
  que va de Marzo a Noviembre, con posibilidad de sesiones extraordinarias en
  Diciembre, Enero y Febrero)

  \section{Estadísticas}\label{section:statistics}
  
  Se obtuvieron estadísticas generales y descriptivas sobre temas de diversa índole en un notebook de Zeppelin, pero el mayor enfoque de este informe se encuentra en las próximas subsecciones:
  
  
  \subsection{Proyectos Presentados por Diputado}
  A continuación, se presentan las estadísticas obtenidas a partir de los proyectos presentados por diputados, con un Top 10 de los diputados con mayor cantidad de proyectos presentados, sancionados y con media sanción, junto con estadísticas descriptivas sobre todos los diputados.
  
  
  \subsection{Proyectos Presentados por Partido}
  
  \subsection{Proyectos Presentados por Distrito}
  
  \subsection{Proyectos Presentados por Comisión de Diputados}
  
  \subsection{Proyectos Presentados por Comisión de Senadores}
  
  \subsection{Conclusiones}
    A partir de lo visualizado en esta sección, se puede concluir lo siguiente:
    
    \begin{itemize}
    	\item Los Top 10 de proyectos con media sanción y sancionados por diputado están completamente gobernados por el Frente Para la Victoria, lo cual es consistente con la composición de la camara de diputados para el período sobre el cual se obtuvieron los datos.
    	
    	\item
    \end{itemize}
  

  \section{Conclusiones Generales}\label{section:conclusions}
  
  % \bibliographystyle{apalike}
  % \bibliography{/Users/crscardellino/Documents/Posgrado/Bibliography/bibliography.bib}

\end{document}
